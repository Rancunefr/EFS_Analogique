\chapter{ Analyse temporelle }

\section{Charge d'un condensateur}

Afin d'étudier la charge d'un condensateur, on utilise le montage RC suivant :

\begin{center}
\includesvg{part01/chap05/charge_c}
\end{center}

À l'instant $t=0$, la tension $U_{in}$ passe de 0V à $E_0$. \\

La loi des mailles donne :

$$U_{in} = U_R + U_C $$

Avec pour la résistance et pour le condensateur :

$$U_R = R\,I$$
$$ I = C \, \dfrac{d\,U_C}{dt} $$

Ce qui permet d'établir l'équation différentielle suivante pour $t>0$~:

$$ RC\,\dfrac{dU_C}{dt} + U_C = E_0 $$ \\

On introduit dans cette expression la constante de temps $\tau = RC$

La solution de l'équation différentielle (en tenant compte des conditions aux limites) est alors~:

$$ U_C(t) = E_0\,(\,1 - e^{-t/\tau}\,) $$

\begin{figure}[!h]
\begin{minipage}{13cm}
\begin{center}
\begin{gnuplot}[terminal=epslatex, terminaloptions=color dashed]
set key at 3,2 horizontal center
set key width 2
set key opaque
set sample 1000
set arrow from graph 0,1 to graph 0,1.05 size screen 0.025,15,60 filled ls -1
set xzeroaxis linetype -1
set xtics axis 
set ytics border nomirror
set border 2
set xr [-1.5:5]
set yr [-1:10]
set xlabel 't (s)'
set ylabel 'Volts'
set tmargin 1
set rmargin 2
R = 2000
C = 400e-6
Tau = R*C
E = 8
step(x) = x>0 ? E : 0
reponse(x) = x<0 ? 0 : E*(1-exp(-x/Tau)) 
plot step(x) w l lc 1 lw 3 t '$U_{in}$',\
reponse(x) w l lc 2 lw 3 t '$U_C$'
\end{gnuplot}
\end{center}
\end{minipage}
\begin{minipage}{3cm}
$E_0 = 8V$ \\
$R = 2\,k\Omega$ \\
$C = 400 \, \mu F$ \\
\bigskip
$ \tau = 0.8\,s $ \\
\end{minipage} \\
	\caption{Charge d'un condensateur}
\end{figure}

\subsection*{Influence de la constante de temps}
\smallskip
\begin{itemize}

\item Au bout d'un temps $3\tau$, la charge du condensateur est à $95\%$ de la valeur finale.

$$ U(3\tau) = E_0\,(\,1-e^{-3}_,) = 0.95 \, E_0$$ \\

\item Au bout d'un temps $5\tau$, la charge du condensateur est à $99.3\%$ de la valeur finale.

$$ U(5\tau) = E_0\,(\,1-e^{-5}_,) = 0.993 \, E_0$$ \\


\item La tangente à l'origine de la courbe coupe la valeur limite ($E_0$) en $t = \tau$

\end{itemize}

\pagebreak

\section{Décharge d'un condensateur}

Afin d'étudier la décharge d'un condensateur, on utilise le montage RC suivant :
\begin{center}
\includesvg{part01/chap05/decharge_c}
\end{center}
À l'instant $t=0$, la tension $U_{in}$ passe de $E_0$ à $0V$. La loi des mailles donne :
\begin{center}
$U_{in} = U_R + U_C$ avec $U_R = R\,I$ et $ I = C \, \dfrac{d\,U_C}{dt} $
\end{center}
Ce qui permet d'établir l'équation différentielle suivante pour $t>0$~:
$$ RC\,\dfrac{dU_C}{dt} + U_C = 0 $$ 
En posant $\tau = RC$, la solution de l'équation différentielle est alors~: $ U_C(t) = E_0\,e^{-t/\tau} $

\begin{figure}[!h]
\begin{minipage}{13cm}
\begin{center}
\begin{gnuplot}[terminal=epslatex, terminaloptions=color dashed]
set key at 3,2 horizontal center
set key width 2
set key opaque
set sample 1000
set arrow from graph 0,1 to graph 0,1.05 size screen 0.025,15,60 filled ls -1
set xzeroaxis linetype -1
set xtics axis 
set ytics border nomirror
set border 2
set xr [-1.5:5]
set yr [-1:10]
set xlabel 't (s)'
set ylabel 'Volts'
set tmargin 1
set rmargin 2
R = 2000
C = 400e-6
Tau = R*C
E = 8
step(x) = x>0 ? 0 : E
reponse(x) = x<0 ? E : E*exp(-x/Tau)
plot step(x) w l lc 1 lw 3 t '$U_{in}$',\
reponse(x) w l lc 2 lw 3 t '$U_C$'
\end{gnuplot}
\end{center}
\end{minipage}
\begin{minipage}{3cm}
$E_0 = 8V$ \\
$R = 2\,k\Omega$ \\
$C = 400 \, \mu F$ \\
\bigskip
$ \tau = 0.8\,s $ \\
\end{minipage} \\
	\caption{décharge d'un condensateur}
\end{figure}

\subsection*{Influence de la constante de temps}
\smallskip
\begin{itemize}

\item Au bout d'un temps $3\tau$, la charge du condensateur est à $5\%$ de la valeur initiale.

$$ U(3\tau) = E_0\,e^{-3} = 0.05 \, E_0$$ \\

\item Au bout d'un temps $5\tau$, la charge du condensateur est à $0.6\%$ de la valeur initiale.

$$ U(5\tau) = E_0\,e^{-5} = 0.006 \, E_0$$ \\

\item La tangente à l'origine de la courbe coupe la valeur 0 en $t = \tau$

\end{itemize}

\section{Établissement du courant dans une inductance}

On considère le schéma du circuit RL suivant :

\begin{center}
\includesvg{part01/chap05/echelon_l}
\end{center}

À l'instant $t=0$, la tension $U_{in}$ passe de 0V à $E_0$. \\

La loi des mailles donne :

$$U_{in} = U_R + U_L $$

Avec pour la résistance et pour le condensateur :

$$U_R = R\,I$$
$$ U_L = L \, \dfrac{d\,I}{dt} $$

Ce qui permet d'établir l'équation différentielle suivante pour $t>0$~:

$$ R\,I + L\,\dfrac{d\,I}{dt}   = E_0 $$ \\

On introduit dans cette expression la constante de temps $\tau = \dfrac{L}{R}$~:

$$ I + \dfrac{L}{R}\dfrac{d\,I}{dt}   = \dfrac{E_0}{R} $$ \\

Pour $t>0$, la solution de cette équation différentielle est~:

$$ I(t) = \dfrac{E_0}{R}\,(\,1 - e^{-t/\tau}\,) $$

Ce qui correspond à la tension $U_L$ suivante :

$$ U_L(t) = L \dfrac{d\,I}{d\,t} = E_0 \, e^{-t/\tau}\ $$


\begin{figure}[!h]
	\begin{center}
	\begin{gnuplot}[terminal=epslatex, terminaloptions={color dashed}]
set key horizontal outside bottom
set key width 2
set key opaque
set sample 1000
set arrow from graph 0,1 to graph 0,1.05 size screen 0.025,15,60 filled ls -1
set arrow from graph 1,1 to graph 1,1.05 size screen 0.025,15,60 filled ls -1
set xzeroaxis linetype -1
set xtics axis 
set ytics  border nomirror
set y2tics border nomirror
set border 10
set xr [-0.5:5.9]
set link y2 via y/10 inverse y*10
set yr [-1:10]
set xlabel 't (s)'
set ylabel 'Volts'
set y2label 'Ampères'
set tmargin 1
set rmargin 2
R = 8
L = 10
Tau = L/R
E = 8
step(x) = x>0 ? E : 0
tension(x) = x<0 ? 0 : E*exp(-x/Tau)
courant(x) = x<0 ? 0 : E/R*(1-exp(-x/Tau)) 
plot step(x) w l lc 1 lw 3 t '$U_{in}$',\
tension(x) w l lc 2 lw 3 t '$U_L$',\
courant(x) w l lc 3 lw 3 t '$I_L$' axis x1y2
\end{gnuplot} 
\begin{minipage}{\textwidth}
\bigskip
\begin{center}
$E_0 = 8V \quad R = 8\,\Omega \quad L = 10\,H \quad \tau = 1.25\,s $ 
\end{center}
\end{minipage}
\end{center}
\caption{Etablissement du courant dans un circuit RL}
\end{figure}

Les équations différentielles étant les mêmes, les remarques concernant la constante de temps $\tau$ effectuées dans les chapitres précédants restent valides. \\


\section{Rupture du courant dans une inductance}

On considère le circuit RL suivant :

\begin{center}
\includesvg{part01/chap05/echelon2_l}
\end{center}

À l'instant $t=0$, la tension $U_{in}$ passe de $E_0$ à $0V$. \\

La loi des mailles donne :

$$U_{in} = U_R + U_L $$

Avec pour la résistance et pour le condensateur :

$$U_R = R\,I$$
$$ U_L = L \, \dfrac{d\,I}{dt} $$ \\

Ce qui permet d'établir l'équation différentielle suivante pour $t>0$~:

$$ R\,I + L\,\dfrac{d\,I}{dt}   = 0 $$ 

On introduit dans cette expression la constante de temps $\tau = \dfrac{L}{R}$~:

$$ I + \dfrac{L}{R}\dfrac{d\,I}{dt}   = 0 $$ \\

Pour $t>0$, la solution de cette équation différentielle est~:

$$ I(t) = \dfrac{E_0}{R}\,e^{-t/\tau} $$  

\pagebreak

Ce qui correspond à la tension $U_L$ suivante :

$$ U_L(t) = L \dfrac{d\,I}{d\,t} = - E_0 \, e^{-t/\tau}\ $$

\begin{figure}[!h]
	\begin{center}
	\begin{gnuplot}[terminal=epslatex, terminaloptions={color dashed}]
set key horizontal outside bottom
set key width 2
set key opaque
set sample 1000
set arrow from graph 0,1 to graph 0,1.05 size screen 0.025,15,60 filled ls -1
set arrow from graph 1,1 to graph 1,1.05 size screen 0.025,15,60 filled ls -1
set xzeroaxis linetype -1
set xtics axis 
set ytics  border nomirror
set y2tics border nomirror
set border 10
set xr [-0.5:5.9]
set link y2 via y/10 inverse y*10
set yr [-10:10]
set xlabel 't (s)'
set ylabel 'Volts'
set y2label 'Ampères'
set tmargin 1
set rmargin 2
R = 8
L = 10
Tau = L/R
E = 8
step(x) = x>0 ? 0 : E
tension(x) = x<0 ? 0 : -E*exp(-x/Tau)
courant(x) = x<0 ? E/R : E/R*exp(-x/Tau)
plot step(x) w l lc 1 lw 3 t '$U_{in}$',\
tension(x) w l lc 2 lw 3 t '$U_L$',\
courant(x) w l lc 3 lw 3 t '$I_L$' axis x1y2
\end{gnuplot} 
\begin{minipage}{\textwidth}
\bigskip
\begin{center}
$E_0 = 8V \quad R = 8\,\Omega \quad L = 10\,H \quad \tau = 1.25\,s $ 
\end{center}
\end{minipage}
\end{center}
\caption{Rupture du courant dans un circuit RL}
\end{figure}

Les équations différentielles étant les mêmes, les remarques concernant la constante de temps $\tau$ effectuées dans les chapitres précédants restent valides. \\

\subsection*{Note:}

Dans le cas présenté ci-dessus, le courant n'est pas brutalement interrompu. Considérer $U_{in} = 0V$ signifie que l'on court-circuite l'alimentation, et donc qu'un chemin est disponible pour qu'un courant puisse s'établir. \\

Lorsque ce n'est pas le cas (ouverture du circuit), le courant passe brutalement de $E_0 / R$ à $0$. Sa dérivée en $t=0$ est donc très grande (théoriquement infinie). La tension aux bornes de l'inductance peut alors s'avérer très (voir trop) importante.  Nous verrons dans les chapitres suivants que cela justifie l'usage d'une diode de roue libre.\\

\section{Réponse à un échelon de tension (RLC série)}

On considère le circuit RLC suivant~:

\begin{center}
\includesvg{part01/chap05/echelon_RLC_Serie}
\end{center}

La loi des mailles donne la relation suivante :

$$ U_{in} = U_R + U_L + U_C $$

Avec pour la résistance, l'inductance et pour la capacité :

$$ U_R = R\,I \quad U_L = L\,\dfrac{d\,I}{d\,t} \quad I = C \, \dfrac{d\,U_C}{d\,t}$$ 

Pour $t>0$, on obtient l'équation différentielle suivante~:

$$ LC\,\dfrac{d^{2}\,U_C}{d\,t^2} + RC\,\dfrac{d\,U_C}{d\,t} + U_C = E_0 $$ \\

On définit alors les constantes suivantes~:\\

\begin{itemize}
\item \textbf{La pulsation propre} ($\omega_0$)

	$$ \omega_0 = \dfrac{1}{\sqrt{LC}} $$

\item \textbf{Le coefficient d'amortissement} ($\alpha$)

	$$ \alpha = \dfrac{R}{2\,L\,\omega_0} = \dfrac{R}{2}\sqrt{\dfrac{C}{L}}$$

\end{itemize}

L'équation du circuit devient alors~:

$$ \dfrac{d^{2}\,U_C}{d\,t^2} + 2\,\alpha\,\omega_0\,\dfrac{d\,U_C}{d\,t} + \omega_0^2\,U_C = \omega_0^2\,E_0 $$\\

On utilise la méthode de résolution d'une EDO d'ordre 2 présentée en \autoref{annexe:edo2}.

\begin{enumerate}

\item \textbf{Solution particulière~:} 
	$$U_C(t) = constante = E_0$$ 

\item \textbf{Solution de l'équation sans second membre~:}

$$ \dfrac{d^{2}\,U_C}{d\,t^2} + 2\,\alpha\,\omega_0\,\dfrac{d\,U_C}{d\,t} + \omega_0^2\,U_C = 0 $$

		Calcul du déterminant~: $\Delta = (2\alpha\omega_0)^2 - 4 * \omega_0^2 = 4\, \omega_0^2\,(\alpha^2 - 1 )$ 

	Le signe de $\Delta $ dépend de $\alpha $. Il y a alors 3 cas possibles~:\\

	\begin{itemize}
		\item \textbf{Le régime apériodique~:} ($\Delta > 0 \Leftrightarrow \alpha > 1 $) \\

			Le polynôme caractéristique admet deux solutions~:
			$$ r_1 = \dfrac{ -2\,\alpha\,\omega_0 - \sqrt{\Delta}}{ 2 } \quad et \quad	r_2 = \dfrac{ -2\,\alpha\,\omega_0 + \sqrt{\Delta}}{ 2 }  $$

			et la solution de l'équation sans second membre est de la forme~:\\
			\begin{center}
				$U_C(t) = A\,e^{r_1\,t} + B\,e^{r_2\,t}$ (avec A et B deux constantes réelles)
			\end{center}
		\bigskip
		\item \textbf{Le régime critique~:} ($\Delta = 0 \Leftrightarrow \alpha = 1$ )\\	

			Le polynôme caractéristique admet une racine double et la solution de l'équation sans second membre est de la forme~:\\

			\begin{center}
				$U_C(t) = (\,A + B\,t\,)\,e^{ -\omega_0\,t} $ (avec A et B deux constantes réelles.) \\
			\end{center}
					
		\bigskip
		\item \textbf{Le régime pseudo-périodique~:} ($\Delta < 0 \Leftrightarrow \alpha < 1$)\\	

			Le polynôme caractéristique admet deux solutions complexes conjuguées~:
			$$ r_1 =  -\,\alpha\,\omega_0 - j\,\dfrac{\sqrt{-\Delta}}{ 2 } \quad et \quad	r_2 = -\,\alpha\,\omega_0 + j\,\dfrac{\sqrt{-\Delta}}{ 2 }  $$

			La solution de l'équation sans second membre est de la forme~:
				$$U_C(t) = e^{-\alpha\omega_0t}\,\left[A\,cos(\,\underbrace{\omega_0\,\sqrt{1-\alpha^2}}_{\omega\,'}\,t)\,+\,B\,sin(\,\underbrace{\omega_0\,\sqrt{1-\alpha^2}}_{\omega\,'}\,t)\right]$$

				avec A et B deux constantes réelles. 

		Cette forme d'équation correspond à un régime sinusoïdal de pulsation $\omega\,'$ avec amortissement exponentiel. On peut aussi l'écrire sous la forme~:
$$ U_C(t) = A\,e^{-\alpha\omega_0t}\,cos(\,\omega\,'\,t + \phi)$$

	avec $A$ et $\phi$ deux constantes réelles. 
		\end{itemize}


	\item \textbf{Ecriture de la solution générale~:}\\

Pour obtenir la solution générale à l'équation du circuit, on ajoute la solution particulière et la solution de l'équation sans second membre. Les valeurs des constantes A et B sont obtenues par l'étude de la tension $U_C$ et du courant aux instants $t=0$ et $t=\infty$.

\end{enumerate}

\subsection*{Régime apériodique $\alpha > 1$}

La solution générale de l'équation différentielle est de la forme~:

$$U_C(t) = E_0 +  A\,e^{r_1\,t} + B\,e^{r_2\,t}$$

avec~:

$$ r_1 = \dfrac{ -2\,\alpha\,\omega_0 - \sqrt{\Delta}}{ 2 } \quad et \quad	r_2 = \dfrac{ -2\,\alpha\,\omega_0 + \sqrt{\Delta}}{ 2 }  $$ \\

En $t=0$, on sait que $Uc=0$ ~:
\begin{align*}
	& \qquad U_C(0) = 0 \\
	& \Leftrightarrow E_0 + A\,e^{r_1\,0} + B\,e^{r_2\,0} = 0 \\
	& \Leftrightarrow E_0 + A + B = 0 \\
	& \Leftrightarrow A + B = -E_0 
\end{align*}

En $t=0$, on sait que $ I = 0 $ ~:
\begin{align*}
	& \qquad I(0) = 0 \\
	& \Leftrightarrow C\,\dfrac{d\,Uc}{d\,t} = 0 \\
	& \Leftrightarrow C\,(A\,r_1 + B\,r_2) = 0  \\
	& \Leftrightarrow (A\,r_1 + B\,r_2) = 0 \\
	& \Leftrightarrow A = -B \, \dfrac{r_2}{r_1} 
\end{align*}

Ces deux équations nous permettent de déterminer les constantes $A$ et $B$.\\ 

Tout calcul fait, $U_C$ est de la forme~:

$$U_C(t) = E_0\,- \dfrac{E_0}{r_2-r_1}\,(\,r_2\,e^{r_1\,t} - r_1\,e^{r_2\,t}\,)$$

\pagebreak

\begin{figure}[!h]
\begin{minipage}{13cm}
\begin{center}
\begin{gnuplot}[terminal=epslatex, terminaloptions=color dashed]
set key at 1.5,10 horizontal center
set key width 2
set key opaque
set sample 1000
set xzeroaxis linetype -1
set xtics axis 
set ytics border nomirror
set border 2
set xr [-0.48:6]
set yr [-1:10]
set xlabel 't (s)'
set ylabel 'Volts'
set tmargin 1
set rmargin 2
R = 8
L = 1
C = 0.25
E = 8
omega = 1/sqrt(L*C)
alpha = R / (2*L*omega)
delta = 4 * omega**2 * ( alpha**2 - 1 )
r1 = (-2*alpha*omega-sqrt(delta) )/2
r2 = (-2*alpha*omega+sqrt(delta) )/2
step(x) = x<0 ? 0 : E
reponse(x) = x<0 ? 0 : E - E/(r2-r1) *( r2*exp(r1*x) - r1*exp(r2*x))
plot step(x) w l lc 1 lw 3 t '$U_{in}$',\
reponse(x) w l lc 2 lw 3 t '$U_C$'
\end{gnuplot}
\end{center}
\end{minipage}
\begin{minipage}{3cm}
$E_0 = 8V$ \\
\bigskip\\
$R = 8 \Omega$ \\
$L = 1 \, H$ \\
$C = 250 \, mF$ \\
\bigskip\\
$ \omega_0 = 2\,rad\,s^{-1} $ \\
$ \alpha = 2 $ 
\end{minipage} \\
	\caption{Réponse apériodique du circuit RLC série}
\end{figure}

\subsection*{Régime critique $\alpha = 1$}

La solution générale de l'équation différentielle est de la forme~:

			$$U_C(t) = E_0 + (\,A + B\,t\,)\,e^{ -\omega_0\,t} $$

En $t=0$, on a~:
\begin{align*}
	& U_C(0) = 0 \\
	& \Leftrightarrow E_0 + A = 0 \\
	& \Leftrightarrow A = - E_0  \\
	& \\
	& I(0) = 0 \\
	& \Leftrightarrow C\,\dfrac{d\,Uc}{d\,t} = 0 \\
	& \Leftrightarrow C ( B - A\,\omega_0 ) = 0 \\
	& \Leftrightarrow B = A \, \omega_0 \\
	& \Rightarrow B = -E_0\, \omega_0 
\end{align*}

On obtient donc~: 

	$$U_C(t) = E_0 - E_0\,(1 + \omega_0\,t\,)\,e^{ -\omega_0\,t} $$


\begin{figure}[!h]
\begin{minipage}{13cm}
\begin{center}
\begin{gnuplot}[terminal=epslatex, terminaloptions=color dashed]
set key at 1.5,10 horizontal center
set key width 2
set key opaque
set sample 1000
set xzeroaxis linetype -1
set xtics axis 
set ytics border nomirror
set border 2
set xr [-0.48:6]
set yr [-1:10]
set xlabel 't (s)'
set ylabel 'Volts'
set tmargin 1
set rmargin 2
R = 4
L = 1
C = 0.25
E = 8
omega = 1/sqrt(L*C)
alpha = R / (2*L*omega)
delta = 4 * omega**2 * ( alpha**2 - 1 )
step(x) = x<0 ? 0 : E
	reponse(x) = x<0 ? 0 : E - E*(1 + omega * x ) * exp(-omega*x)
plot step(x) w l lc 1 lw 3 t '$U_{in}$',\
reponse(x) w l lc 2 lw 3 t '$U_C$'
\end{gnuplot}
\end{center}
\end{minipage}
\begin{minipage}{3cm}
$E_0 = 8V$ \\
\bigskip\\
$R = 4 \Omega$ \\
$L = 1 \, H$ \\
$C = 250 \, mF$ \\
\bigskip\\
$ \omega_0 = 2 \,rad\,s^{-1} $ \\
$ \alpha = 1,125 $ 
\end{minipage} \\
	\caption{Réponse critique du circuit RLC série}
\end{figure}


\subsection*{Régime pseudo-périodique $\alpha < 1$}

La solution générale de l'équation différentielle est~:

$$U_C(t) = E_0 + e^{-\alpha\omega_0t}\,\left[A\,cos(\,\underbrace{\omega_0\,\sqrt{1-\alpha^2}}_{\omega\,'}\,t)\,+\,B\,sin(\,\underbrace{\omega_0\,\sqrt{1-\alpha^2}}_{\omega\,'}\,t)\right]$$

en $t=0$~:
\begin{align*}
	& U_C(0) = 0 \\
	& \Leftrightarrow E_0 + A = 0 \\
	& \Leftrightarrow A = -E_0 \\
	& \\
	& I(0) = 0 \\
	& \Leftrightarrow C\,\dfrac{d\,Uc}{d\,t} = 0 \\
	& \Leftrightarrow \dfrac{d\,Uc}{d\,t} = 0 \\
	& \Leftrightarrow B = \alpha \dfrac{\omega_0}{\omega\,'} A 
\end{align*}

\pagebreak

$U_C$ s'écrit alors~:

$$U_C(t) = E_0 -E_0\,e^{-\alpha\omega_0t}\,\left[\,cos(\,\omega\,'\,t)\,+\,\dfrac{\alpha\,\omega_0}{\omega\,'}\,sin(\,\omega\,'\,t)\,\right]$$ \\

\begin{figure}[!h]
\begin{minipage}{13cm}
\begin{center}
\begin{gnuplot}[terminal=epslatex, terminaloptions=color dashed]
set key at 6,10 horizontal center
set key width 2
set key opaque
set sample 1000
set xzeroaxis linetype -1
set xtics axis 
set ytics border nomirror
set border 2
set xr [-0.48:8]
set yr [-1:12]
set xlabel 't (s)'
set ylabel 'Volts'
set tmargin 1
set rmargin 2
R = 1
L = 1
C = 0.25
E = 8
omega = 1/sqrt(L*C)
alpha = R / (2*L*omega)
delta = 4 * omega**2 * ( alpha**2 - 1 )
puls = omega * sqrt( 1-alpha**2)
step(x) = x<0 ? 0 : E
reponse(x) = x<0 ? 0 : E - E * exp( - alpha * omega * x ) * ( cos(puls*x) + alpha * omega / puls * sin(puls*x))
plot step(x) w l lc 1 lw 3 t '$U_{in}$',\
reponse(x) w l lc 2 lw 3 t '$U_C$'
\end{gnuplot}
\end{center}
\end{minipage}
\begin{minipage}{3cm}
$E_0 = 8V$ \\
\bigskip\\
$R = 1 \Omega$ \\
$L = 1 \, H$ \\
$C = 250 \, mF$ \\
\bigskip\\
$ \omega_0 = 2 \,rad\,s^{-1} $ \\
$ \omega\,' = 1.9\,rad\,s^{-1} $ \\
$ \alpha = 0.25 $ 
\end{minipage} \\
	\caption{Réponse pseudo-périodique du circuit RLC série}
\end{figure}


L'expression de $U_C$ se décompose en deux parties : \\
\begin{itemize}
\item Une partie oscillante à la pulsation $\omega\,'$
\item Une amplitude décroissance de manière exponentielle.
\end{itemize}

\begin{center}
\begin{gnuplot}[terminal=epslatex, terminaloptions={color dashed size 10cm,6cm}]
set key off
set sample 1000
set xzeroaxis linetype -1
set xtics axis 
set ytics border nomirror
set border 2
set xr [-0.1:8]
set yr [0:16]
set xlabel 't (s)'
set ylabel 'Volts'
set tmargin 1
set rmargin 2
R = 1
L = 1
C = 0.25
E = 8
omega = 1/sqrt(L*C)
alpha = R / (2*L*omega)
delta = 4 * omega**2 * ( alpha**2 - 1 )
puls = omega * sqrt( 1-alpha**2)
step(x) = x<0 ? 0 : E
reponse(x) = x<0 ? 0 : E - E * exp( - alpha * omega * x ) * ( cos(puls*x) + alpha * omega / puls * sin(puls*x))
sup(x) =  E - E * exp( - alpha * omega * x )*1.05
inf(x) =  E + E * exp( - alpha * omega * x )*1.05 
plot sup(x) w l lc 3 lw 3 t '$U_{in}$',\
inf(x) w l lc 3 lw 3 t '$U_{in}$',\
reponse(x) w l lc 2 lw 3 t '$U_C$'
\end{gnuplot}
\end{center}

\subsection*{Influence du coefficient d'amortissement}

Comme on pu le constater ci-dessus, le coefficient d'amortissement $\alpha$ détermine à lui seul le type de réponse du circuit RLC~: \\

\begin{itemize}
\item Pour $\alpha > 1$, le régime est apériodique. 
\item Pour $\alpha < 1$, le régime est pseudo-périodique. \\
\end{itemize}

La limite entre ces deux comportements est le régime critique. ( $\alpha = 1 $ ) \\

\begin{figure}[!h]
\begin{center}
\begin{gnuplot}[terminal=epslatex, terminaloptions={color dashed size 15cm,6cm}]
set key at screen 1, graph 1
set sample 1000
set xzeroaxis linetype -1
set xtics axis 
set ytics border nomirror
set border 2
set xr [-0.9:8]
set yr [0:14]
set xlabel 't (s)'
set ylabel 'Volts'
set tmargin 1
set rmargin 15
L = 1
C = 0.25
E = 8
omega = 1/sqrt(L*C)
delta(X) = 4*omega**2*(X**2-1)
	om(Y) = omega*sqrt(1-Y**2) 
r2(X) = -X*omega+sqrt(delta(X))/2
r1(X) = -X*omega-sqrt(delta(X))/2
Uap(x,Y) = x<0 ? 0 : E - E/(r2(Y)-r1(Y))*(r2(Y)*exp(r1(Y)*x)-r1(Y)*exp(r2(Y)*x))
Uc(x) = x<0 ? 0 : E -E*(1+omega*x)*exp(-omega*x)
	Upe(x,Y) = x<0 ? 0 : E - E*exp(-Y*omega*x)*(cos(om(Y)*x)+(Y*omega/om(Y))*sin(om(Y)*x))
plot \
	 Uap(x,3.0) w l lc 5 lw 5 t '$\alpha = 3.0$', \
	 Uap(x,2.0) w l lc 5 lw 3 t '$2.0$', \
	 Uap(x,1.5) w l lc 5 lw 3 t '$1.5$', \
	 Uap(x,1.2) w l lc 5 lw 2 t '$1.2$', \
	 Uc(x) w l lc 1 lw 4 t '$1.0$', \
	 Upe(x,0.8) w l lc 3 lw 2 t '$0.8$', \
	 Upe(x,0.6) w l lc 3 lw 3 t '$0.6$', \
	 Upe(x,0.4) w l lc 3 lw 4 t '$0.4$', \
	 Upe(x,0.2) w l lc 3 lw 5 t '$0.2$'
\end{gnuplot}
\end{center}
\end{figure}

Pour $\alpha=0$, il n'y a aucun amortissement~: les oscillations sont entretenues. Pour $\alpha<0$, le système les oscillations divergent car le système gagne de l'énergie~: \\

\begin{figure}[!h]
\begin{center}
\begin{gnuplot}[terminal=epslatex, terminaloptions={color dashed size 15cm,7cm}]
set key at screen 1, graph 1
set sample 1000
set xzeroaxis linetype -1
set xtics border nomirror
set ytics border nomirror
set border 3
set xr [-0.9:8]
set yr [-50:50]
set xlabel 't (s)'
set ylabel 'Volts'
set tmargin 1
set rmargin 15
L = 1
C = 0.25
E = 8
omega = 1/sqrt(L*C)
delta(X) = 4*omega**2*(X**2-1)
	om(Y) = omega*sqrt(1-Y**2) 
r2(X) = -X*omega+sqrt(delta(X))/2
r1(X) = -X*omega-sqrt(delta(X))/2
Uap(x,Y) = x<0 ? 0 : E - E/(r2(Y)-r1(Y))*(r2(Y)*exp(r1(Y)*x)-r1(Y)*exp(r2(Y)*x))
Uc(x) = x<0 ? 0 : E -E*(1+omega*x)*exp(-omega*x)
	Upe(x,Y) = x<0 ? 0 : E - E*exp(-Y*omega*x)*(cos(om(Y)*x)+(Y*omega/om(Y))*sin(om(Y)*x))
plot \
	 Upe(x,0.0)   w l lc 2 lw 6 t '$\alpha = 0.0$', \
	 Upe(x,-0.1)  w l lc 4 lw 3 t '$-0.1$', \
	 Upe(x,-0.15) w l lc 7 lw 3 t '$-0.15$'
\end{gnuplot}
\end{center}
\end{figure}

\section{Réponse à un échelon de courant (RLC parallèle)}

On considère le circuit RLC parallèle suivant~:

\begin{center}
\includesvg{part01/chap05/echelon_RLC_Parallele}
\end{center}

La loi des mailles nous donne~:

$$ I = I_R + I_L + I_C $$

Avec pour la résistance, l'inductance et pour la capacité :

$$ U = R\,I_R \quad U = L\,\dfrac{d\,I_L}{d\,t} \quad I_C = C \, \dfrac{d\,U}{d\,t}$$  \\


Pour $t>0$, on obtient l'équation différentielle suivante~:

$$ \dfrac{d\,I}{d\,t} =  C\,\dfrac{d^2\,U}{d\,t^2} + \dfrac{1}{R} \dfrac{d\,U}{d\,t} +  \dfrac{1}{L}\,U $$ \\

On retrouve ici une EDO d'ordre deux, tout à fait similaire à celle obtenue pour le circuit RLC série étudié dans la section précédente. \\

Ce circuit porte le nom de "\textbf{circuit bouchon}" car il est couramment utilisé en bout de ligne de transmission. Nous y reviendrons un peu plus tard. \\


