\chapter{Rappels de trigonométrie}

\begin{center}
\includesvg{annexes/annexe_02/cercle}
\end{center}

Le cercle trigonométrique est un cercle \textbf{\underline{unitaire}}. \\

On peut donc voir que~: \\
$$ -1 \leq cos(\theta) \leq 1 $$
$$ -1 \leq sin(\theta) \leq 1 $$

\pagebreak
\section*{Valeurs remarquables}
\begin{minipage}{0.5\textwidth}
\begin{center}
\includesvg{annexes/annexe_02/cercle2}
\end{center}
\end{minipage}
\begin{minipage}{0.5\textwidth}
\begin{center}
	\setlength{\tabulinesep}{2mm}
	\begin{tabu} to 80mm {|X[c]|X[c]|X[c]|}
	\hline
	$\bm{\theta}$ & $\bm{cos(\theta)}$ & $\bm{sin(\theta)}$ \\ 
	\hline
	$0$ & $1$ & $0$ \\ 
	\hline
	$\dfrac{\pi}{6}$ & $\dfrac{\sqrt{3}}{2}$ & $\dfrac{1}{2}$ \\ 
	\hline
		$\dfrac{\pi}{4}$ & $\dfrac{\sqrt{2}}{2}$ & $\dfrac{\sqrt{2}}{2}$ \\ 
	\hline
		$\dfrac{\pi}{3}$ & $\dfrac{1}{2}$ & $\dfrac{\sqrt{3}}{2}$ \\ 
	\hline
	$\dfrac{\pi}{2}$ & $0$ & $1$ \\ 
	\hline
\end{tabu}
\end{center}
\end{minipage}
\section*{Relations dans le cercle}

\vfill
\begin{minipage}{0.5\textwidth}
\begin{center}
\includesvg{annexes/annexe_02/cercle3}
\end{center}
\end{minipage}
\begin{minipage}{0.5\textwidth}
\begin{tabular}{l}
	$cos(-\theta) = cos(\theta)$ \\
	$sin(-\theta) = -sin(\theta)$ \\
	\\
	$cos(\pi-\theta) = -cos(\theta)$ \\
	$sin(\pi-\theta) = sin(\theta)$ \\
	\\
	$cos(\pi+\theta) = -cos(\theta)$ \\
	$sin(\pi+\theta) = -sin(\theta)$ \\
\end{tabular}
\end{minipage} 

\vfill

\begin{minipage}{0.5\textwidth}
\begin{center}
	\hspace{0.5cm}\includesvg{annexes/annexe_02/cercle4}
\end{center}
\end{minipage}
\begin{minipage}{0.5\textwidth}
	\begin{tabular}{l}
	$cos(\theta+\frac{\pi}{2}) = sin(\theta)$ \\
	$sin(\theta+\frac{\pi}{2}) = -cos(\theta)$ \\
	\\
	$cos(\frac{\pi}{2}-\theta) = sin(\theta)$ \\
	$sin(\frac{\pi}{2}-\theta) = cos(\theta)$ \\
\end{tabular}
\end{minipage}

\pagebreak

\section*{Formulaire}

	\begin{tabular}{l}

		$tan(a) = \dfrac{sin(a)}{cos(a)}$ \\
		\\
		$cos^2(a) + sin^2(a) = 1$ \\
		\\
		$cos(a+b) = cos(a)cos(b)-sin(a)sin(b)$ \\
		$cos(a-b) = cos(a)cos(b)+sin(a)sin(b)$ \\
		\\
		$sin(a+b) = sin(a)cos(b)+cos(a)sin(b)$ \\
		$sin(a-b) = sin(a)cos(b)-cos(a)sin(b)$ \\
\end{tabular}

\vfill

\section*{Formule de Moivre}

$$ (e^{i\theta})^n = e^{i\,n\,\theta} $$

\section*{Formule d'Euler}

\begin{center}
	$$ cos(\theta) = \dfrac{e^{i\theta}+e^{-i\theta}}{2} $$
	$$ sin(\theta) = \dfrac{e^{i\theta}-e^{-i\theta}}{2\,i} $$
\end{center}

\section*{Exemple de linéarisation}

\begin{equation*}
	\setlength{\jot}{10pt}
	\begin{split}
	cos^4(\theta) & = \left(\dfrac{e^{i\theta}+e^{-i\theta}}{2}\right)^4\\
				  & = \dfrac{ e^{i4\theta} + 4e^{3i\theta}e^{-i\theta} + 6ei^{2i\theta}e^{-2i\theta} + 4e^{i\theta}e^{-3i\theta} + e^{-4i\theta}}{16} \\
				  & = \dfrac{1}{8}\left(\dfrac{ e^{i4\theta} +  e^{-i4\theta}}{2}\right) +  \dfrac{4}{8}\left(\dfrac{e^{i2\theta}+e^{-i2\theta}}{2}\right) + \dfrac{6}{16}\\
				  & = \dfrac{1}{8}\,cos(4\theta) + \dfrac{1}{2}\,cos(2\theta) + \dfrac{3}{8} \\
	\end{split}
\end{equation*}

\vfill

\pagebreak

\section*{Courbes des fonctions Sinus et Cosinus}


\begin{figure}[!h]
\begin{minipage}{0.5\textwidth}
\begin{center}
	\begin{gnuplot}[terminal=epslatex, terminaloptions={color dashed size 8cm,6cm} ]
# Legend
set key at 6.1,1.4
# Axes ranges
set xrange [-2*pi:2*pi]
set yrange [-1.5:1.5]
# Axes tics
set xzeroaxis
set yzeroaxis
set xtics axis ('$-2\pi$' -2*pi,'$-\pi$' -pi,0,'$\pi$' pi,'$2\pi$' 2*pi)
set ytics axis 1
set tics scale 0.75
set border 0
# Line width of the axes
set border linewidth 1.5
# Line styles
set style line 1 linecolor rgb '#0060ad' linetype 1 linewidth 3
set style line 2 linecolor rgb '#dd181f' linetype 1 linewidth 3
plot cos(x) title 'cos(x)' with lines linestyle 2
\end{gnuplot}
\end{center}
\end{minipage}
	\begin{minipage}{0.5\textwidth}
	La fonction $cos(x)$ est $2\pi$ périodique. \\

	La fonction $cos(x)$ est \textbf{paire}. \\
	
	\end{minipage}
\end{figure}

\begin{figure}[!h]
\begin{minipage}{0.5\textwidth}
\begin{center}
	\begin{gnuplot}[terminal=epslatex, terminaloptions={color dashed size 8cm,6cm} ]
# Legend
set key at 6.1,1.4
# Axes ranges
set xrange [-2*pi:2*pi]
set yrange [-1.5:1.5]
# Axes tics
set xzeroaxis
set yzeroaxis
set xtics axis ('$-2\pi$' -2*pi,'$-\pi$' -pi,0,'$\pi$' pi,'$2\pi$' 2*pi)
set ytics axis 1
set tics scale 0.75
set border 0
# Line width of the axes
set border linewidth 1.5
# Line styles
set style line 1 linecolor rgb '#0060ad' linetype 1 linewidth 3
set style line 2 linecolor rgb '#dd181f' linetype 1 linewidth 3
plot sin(x) title 'sin(x)' with lines linestyle 1
\end{gnuplot}
\end{center}
\end{minipage}
\begin{minipage}{0.5\textwidth}
	La fonction $sin(x)$ est $2\pi$ périodique. \\

	La fonction $sin(x)$ est \textbf{impaire}. \\
	
	\end{minipage}

\end{figure}

\begin{figure}[!h]
\begin{minipage}{0.5\textwidth}
\begin{center}
	\begin{gnuplot}[terminal=epslatex, terminaloptions={color dashed size 8cm,6cm} ]
# Legend
set key at 6.1,1.4
# Axes ranges
set xrange [-2*pi:2*pi]
set yrange [-1.5:1.5]
# Axes tics
set xzeroaxis
set yzeroaxis
set xtics axis ('$-2\pi$' -2*pi,'$-\pi$' -pi,0,'$\pi$' pi,'$2\pi$' 2*pi)
set ytics axis 1
set tics scale 0.75
set border 0
# Line width of the axes
set border linewidth 1.5
# Line styles
set style line 1 linecolor rgb '#0060ad' linetype 1 linewidth 3
set style line 2 linecolor rgb '#dd181f' linetype 1 linewidth 3
set style line 3 linecolor rgb '#1fdd1f' linetype 1 linewidth 3
plot tan(x) title 'tan(x)' with lines linestyle 3
\end{gnuplot}
\end{center}
\end{minipage}
\begin{minipage}{0.5\textwidth}
	La fonction $tan(x)$ est $\pi$ périodique. \\

	La fonction $tan(x)$ est \textbf{impaire}. \\
	
	\end{minipage}

\end{figure}
