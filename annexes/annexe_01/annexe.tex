\chapter{Les unités}

\section{Les unités S.I.}

Les septs unités du système SI sont les unités à partir desquelles on peut contruire toutes autres unités en physiques. Ces sept unités de bases sont les suivantes~:

\begin{center}
\bgroup
\def\arraystretch{1.2}%  1 is the default, change whatever you need
\rowcolors{2}{blue!30}{blue!10}
\begin{tabular}{|c c c|}
	\hline
	\textbf{Symbole} & \textbf{Nom} & \textbf{Usage} \\
	\hline
	\hline
	m & Mètre & Longueur \\
	kg & Kilogramme & Masse \\
	s & Seconde & Temps \\
	A & Ampère & Intensité de Courant électrique \\
	K & Kelvin & Température \\
	cd & Candela & Intensité lumineuse \\
	mol & Mole & Quantité de matière \\
	\hline
\end{tabular}
\egroup
\end{center}

\section{Les unités courantes en electronique}
\begin{center}
\bgroup
\def\arraystretch{1.2}%  1 is the default, change whatever you need
\rowcolors{2}{blue!30}{blue!10}
\begin{tabular}{|c c c c|}
	\hline
	\textbf{Symbole} & \textbf{Nom} & \textbf{Usage} & \textbf{Unités SI}  \\
	\hline
	\hline
	C & Coulomb & Charge & $A\,s$ \\
	F & Farad & Capacité électrique & $m^{-2}\,kg^{-1}\,s^{4}\,A^{2}$ \\
	H & Henry & Inductance & $m^2\,kg\,s^{-2}\,A^{-2}$ \\
	S & Siemens & Conductance, Admittance, Suceptance & $m^{-2}\,kg^{-1}\,s^{3}\,A^{2}$ \\
	J & Joule & Energie, Travail, Quantité de chaleur & $kg\,m^2\,s^{-2}$  \\
	V & Volt & Force electromotrice, Potentiel & $kg\,m^{2}\,s^{-3}\,A^{-1}$  \\
	W & Watt & Puissance, Flux energétique & $kg\,m^2\,s^{-3}$$kg\,m^2\,s^{-3}$ \\
	$\Omega$ & Ohm & Résistance & $m^2\,kg\,s^{-3}\,A^2$ \\
	\hline
\end{tabular}
\egroup
\end{center}

\section{Les multiples}
\smallskip
\begin{center}
\bgroup
\def\arraystretch{1.2}%  1 is the default, change whatever you need
\rowcolors{2}{blue!30}{blue!10}
\begin{tabular}{|p{0.1\textwidth}>{\centering}p{0.1\textwidth}>{\centering\arraybackslash}p{0.1\textwidth}|}
	\hline
	\textbf{Facteur} & \textbf{Nom} & \textbf{Symbole}  \\
	\hline
	\hline
	$10^{-30}$ & quecto & q \\
	$10^{-27}$ & ronto & r \\
	$10^{-24}$ & yocto & y \\
	$10^{-21}$ & zepto & z \\
	$10^{-18}$ & atto & a \\
	$10^{-15}$ & femto & f \\
	$10^{-12}$ & pico & p \\
	$10^{-9}$ & nano & n \\
	$10^{-6}$ & micro & $\mu$ \\
	$10^{-3}$ & milli & m \\
	$10^{-2}$ & centi & c \\
	$10^{-1}$ & deci & d \\
	\textbf{1} & \textbf{unité} & \\
	$10^{1}$ & deca & da \\
	$10^{2}$ & hecto & h \\
	$10^{3}$ & kilo & k \\
	$10^{6}$ & mega & M \\
	$10^{9}$ & giga & G \\
	$10^{12}$ & tera & T \\
	$10^{15}$ & peta & P \\
	$10^{18}$ & exa & E \\
	$10^{21}$ & zetta & Z \\
	$10^{24}$ & yotta & Y \\
	$10^{27}$ & ronna & R \\
	$10^{30}$ & quetta & Q \\
	\hline
\end{tabular}
\egroup
\end{center}

