\documentclass[a4paper]{book}
\usepackage[a4paper,margin=2cm,top=4cm,headsep=20mm]{geometry}
\usepackage{fancyhdr}
\usepackage{lipsum}
\usepackage{tabu}
\fancyhf{}

% Redéfinition de la commande \chapter
\makeatletter

\def\nompart{Le nom de la partie}
\def\nomfiche{Le nom de la fifiche}

\newcommand{\partie}[1]{%
	\def\nompart{#1}
	\part{#1}
    }%
\newcommand{\fiche}[1]{%
	\def\nomfiche{#1}
	\newpage

    }%

\makeatother

\begin{document}

\renewcommand{\headrulewidth}{0pt}
\fancyhead[CO,CE]{
	\setlength{\tabulinesep}{2mm}
	\begin{tabu} to \textwidth {|X[l,6]|X[c,1]|}
	\hline
	\textbf{\nompart} & Fiche 01 \tabularnewline
	\hline
	\multicolumn{2}{|l|}{ \huge \textbf{ \nomfiche }} \tabularnewline
	\hline
\end{tabu}
}

\pagestyle{fancy}
\partie{test de chapitre}

\fiche{Les circuits RC}
\section{coucou}
\lipsum[1-3]
\section{coucou22}

Le nom de la fifiche est "\nompart{}" \\

\end{document}
