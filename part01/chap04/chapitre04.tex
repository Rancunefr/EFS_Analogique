\chapter{Les circuits linéaires}

\index{ciyyrcuit!circuit lineaire@circuit linéaire}
On appelle "\textbf{circuit linéaire}" tout circuit formé uniquement de dipôles linéaires (résistance, capacités et inductances) et de sources de tension ou de courant indépendantes. \\

Les circuits linéaires sont une grande famille de circuits particulièrement faciles à manipuler, car cette propriété de linéarité nous donne des théorèmes très utiles :\\

\begin{itemize}
\item Le théorème de superposition
\item Le théorème de Thévenin
\item Le théorème de Norton
\end{itemize}


\section{Théorème de superposition}

Le \textbf{théorème de superposition}, ou "\textbf{théorème de Helmotz}" est rendu possible par la linéarité des équations régissant cette famille de circuit. Il s'exprime en ces termes~:\\

\textbf{Dans un circuit linéaire avec plusieurs sources, le courant et la tension pour tout élément du circuit est la somme des courants et des tensions produits par chaque source agissant indépendamment.} \\

Pour appliquer ce théorème, on calcule la contribution de chacune des sources de tension ou de courant au résultat final en "éteignant" toutes les autres sources.\\

Pour "éteindre" une source de tension, on la remplace par un fil. \\

Pour "éteindre" une source de courant, on la remplace par un circuit ouvert. \\

On somme ensuite toutes les contributions pour obtenir le résultat final.

\pagebreak
\subsubsection*{Exemple~:}

On cherche la valeur de la tension $U_{AB}$ dans le circuit suivant~:

\begin{center}
	\includesvg[scale=0.8]{part01/chap04/superposition}
\end{center}


\subsubsection*{Calcul de la contribution de $U_G$}


On "éteint" tous les générateurs à l'exception du générateur de tension $U_G$. Le circuit devient alors~:

\begin{center}
	\includesvg[scale=0.8]{part01/chap04/superposition_2}
\end{center}

On peut alors le réarranger afin de le rendre plus facilement lisible~:

\begin{center}
	\includesvg[scale=0.8]{part01/chap04/superposition_3}
\end{center}

Le courant débité par le générateur $U_G$ s'écrit alors :

$$ I = \dfrac{U_G}{R_1 + R_2} $$

Ce courant traversant la résistance $R_2$, on peut alors écrire~:

$$ U_{AB} = R_2 \, I = \dfrac{R_2}{R_1 + R_2} U_G $$

\subsubsection*{Calcul de la contribution de $I_G$}

On "éteint" cette fois-ci tous les générateurs à l'exception du générateur de courant. $U_G$ est donc remplacé par un fil :

\begin{center}
	\includesvg[scale=0.8]{part01/chap04/superposition_4}
\end{center}

Une fois réarrangé un peu, le schéma devient~:

\begin{center}
	\includesvg[scale=0.8]{part01/chap04/superposition_5}
\end{center}

En utilisant la résistance équivalente, il vient alors~:

$$ U_{AB} = R_{eq}\,I_G = \dfrac{R_1 R_2}{R_1 + R_2}\,I_G $$

\subsubsection*{Conclusion}

\begin{center}
	\includesvg[scale=0.8]{part01/chap04/superposition}
\end{center}

Pour obtenir la réponse finale, il suffit alors de sommer les contributions des différentes sources~:

$$ U_{AB} = \underbrace{\dfrac{R_2}{R_1 + R_2} U_G}_{Contribution\,de\,U_G} + \underbrace{\dfrac{R_1 R_2}{R_1 + R_2}\,I_G}_{Contribution\,de\,I_G} $$

\pagebreak

\section{Théorème de Thévenin}

Ce théorème découle des propriétés de linéarité des circuits considérés et donc du théorème de superposition. Il s'énonce de la façon suivante : \\

\textbf{"Un réseau électrique linéaire vu de deux points est équivalent à un générateur de tension parfait dont la force électromotrice est égale à la différence de potentiels à vide entre ces deux points, en série avec une résistance égale à celle que l'on mesure entre les deux points lorsque les générateurs indépendants sont rendus passifs."}


\begin{center}
	\includesvg{part01/chap04/thevenin}
\end{center}

La tension $E_{Th}$ du générateur de Thévenin est égale à la tension à vide du montage, mesurée entre $A$ et $B$.



%\section{Théorème de Norton}

%\section{Equivalence entre Thévenin et Norton}
%\section{Le diviseur de tension}
%\section{Le diviseur de courant}
