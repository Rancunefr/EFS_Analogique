\chapter{Les circuits linéaires}

\index{ciyyrcuit!circuit lineaire@circuit linéaire}
On appelle "\textbf{circuit linéaire}" tout circuit formé uniquement de dipôles linéaires (résistances, générateurs de courant ou de tension, capacités et inductances).

\section{Théorème de superposition}

Le \textbf{théorème de superposition}, ou "\textbf{théorème de Helmotz}" est rendu possible par la linéarité des équations régissant cette famille de circuit. Il s'exprime en ces termes~:\\

\textbf{Dans un circuit linéaire avec plusieurs sources, le courant et la tension pour tout élément du circuit est la somme des courants et des tensions produits par chaque source agissant indépendamment.} \\

Pour appliquer ce théorème, on calcule la contribution de chacune des sources de tension ou de courant au résultat final en "éteignant" toutes les autres sources.\\

Pour "éteindre" une source de tension, on la remplace par un fil. \\

Pour "éteindre" une source de courant, on la remplace par un circuit ouvert. \\

On somme ensuite toutes les contributions pour obtenir le résultat.

\subsubsection*{Exemple~:}

On cherche la valeur de la tension $U_{AB}$ dans le circuit suivant~:

\begin{center}
	\includesvg[scale=0.8]{part01/chap04/superposition}
\end{center}















\section{Théorème de Thévenin}

Ce théorème découle des propriétés de linéarité des circuits considérés et donc du théorème de superposition. Il s'énonce de la façon suivante : \\

\textbf{"Un réseau électrique linéaire vu de deux points est équivalent à un générateur de tension parfait dont la force électromotrice est égale à la différence de potentiels à vide entre ces deux points, en série avec une résistance égale à celle que l'on mesure entre les deux points lorsque les générateurs indépendants sont rendus passifs."}



\section{Théorème de Norton}

\section{Equivalence entre Thévenin et Norton}
\section{Le diviseur de tension}
\section{Le diviseur de courant}
