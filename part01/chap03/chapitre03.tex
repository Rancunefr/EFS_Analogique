\chapter{ Les dipôles idéaux }

Les \textbf{dipôles idéaux} correspondent à des relations entre la tension à leurs bornes et le courant qui les traverse (On parle de "\textbf{caractéristique courant/tension}"). Comme leur nom le laisse supposer, ces dipôles sont des représentations idéales des composants réels. Nous nous en servirons comme briques de base afin de modéliser les circuits. 

\section{Les générateurs idéaux}

\subsection{ Le générateur de tension }

Un \textbf{générateur idéal de tension} est un générateur dont la tension est constante, et ce quel que soit le courant demandé.

\begin{figure}[!h]
\begin{center}
\includesvg[scale=0.7]{part01/chap03/generateur_ideal_tension}
\hspace{1cm}
\includesvg[scale=0.7]{part01/chap03/carac_generateur_ideal_tension}
\end{center}
\caption{ Symbole et caractéristique courant/tension du générateur idéal de tension}
\end{figure}

Le générateur de tension ne peut être que théorique car mis en court-circuit, il devrait délivrer un courant infini et donc fournir au circuit une puissance également infinie.\\

Cette définition du générateur idéal de tension est parfois étendue pour des générateurs dont la tension est une fonction du temps $u(t)$. Dans ce cas, la tension fournie ne dépendra que du temps et pas du courant. 

\subsection{ Le générateur idéal de courant }

Un \textbf{générateur idéal de courant} est un générateur fournissant un courant constant, et ce quel que soit la tension appliquée à ses bornes.

\begin{figure}[!h]
\begin{center}
\includesvg[scale=0.7]{part01/chap03/generateur_ideal_courant}
\hspace{1cm}
\includesvg[scale=0.7]{part01/chap03/carac_generateur_ideal_courant}
\end{center}
\caption{ Symbole et caractéristique courant/tension du générateur idéal de courant}
\end{figure}

Tout comme le générateur idéal de tension, c'est un générateur théorique car dans le cas du circuit ouvert il fournirait une tension infinie.\\

Ici encore, on peut étendre cette définition à des générateurs dont le courant n'est pas constant, mais une fonction du temps $I(t)$. Dans ce cas, le courant fourni ne dépendra que du temps et pas de la tension appliquée aux bornes du générateur.


\section{ Les dipôles linéaires }

On parle de \textbf{dipôles "linéaires"} (ce qui est un petit abus de langage) pour désigner les dipôles possédant une relation linéaire entre~:\\
\begin{itemize}
\item tension et courant, 
\item tension et dérivée du courant,
\item ou courant et dérivée de la tension.\\
\end{itemize}

Ces dipôles linéaires sont au nombre de trois~: \\

\begin{itemize}
\item \textbf{La résistance}
\item \textbf{L'inductance}
\item \textbf{La capacité} \\
\end{itemize}

Il faut bien faire la différence entre ces trois dipôles idéaux et leurs "incarnations" en composants que sont les résistors, les bobines et les condensateurs. \\

\subsection{ La résistance }

\vspace{0.5cm}
\begin{tabular}{ll}
\textbf{Notation usuelle~:} & $R$ \\
\textbf{Unité~:} & Ohm ($\Omega$) \\
\textbf{Unité SI~:} & $m^2 \cdot kg \cdot {s}^{-3} \cdot A^{-2}$ \\
\textbf{Nature~:} & Grandeur scalaire \\
\end{tabular} 

\subsubsection*{Définition}

La \textbf{résistance} traduit une relation linéaire entre courant et tension. Le symbole qui permet de la représenter est généralement l'un des deux suivants~:

\begin{figure}[!h]
\centering
\begin{subfigure}{.4\textwidth}
\centering
	\includesvg[width=0.8\textwidth]{part01/chap03/symbole_r_euro}
\caption{Symbole Euro}
\end{subfigure}%
\begin{subfigure}{.4\textwidth}
\centering
	\includesvg[width=0.8\textwidth]{part01/chap03/symbole_r_us} 
\caption{Symbole US}
\end{subfigure}
\caption{Symbole de la résistance}
\end{figure}

Le comportement courant/tension d'une résistance est défini par la \textbf{loi d'Ohm}~:\\
\begin{center}
\begin{minipage}{.2\textwidth}
\begin{center}
\includesvg{part01/chap03/carac_R}
\end{center}
\end{minipage}
\hspace{1cm}
\begin{minipage}{.3\textwidth} 
\begin{equation}
	I_R = \dfrac{U_R}{R}
\end{equation}
\end{minipage}
\end{center}

avec~:\\

\begin{itemize}
\item $U_R$ La tension aux bornes de la résistance
\item $I_R$ Le courant traversant la résistance
\item $R$ La valeur de la résistance en Ohms\\
\end{itemize}


Lorsqu'un conducteur montre une caractéristique courant/tension vérifiant la loi d'Ohm (une droite passant par l'origine), on parle de "\textbf{Conducteur ohmique}". On utilise parfois également les termes de "\textbf{résistance pure}" ou "\textbf{résistance idéale}".

\subsubsection*{Effet Joule}

Physiquement, le courant est un mouvement de porteurs de charge. Or dans un conducteur ohmique, ces porteurs interagissent avec les atomes constitutifs du milieu dans lequel ils se déplacent, ce qui constitue un frein à leur mouvement. Ceci se traduit par l'\textbf{effet~Joule}. C'est un effet thermique qui provoque une augmentation de l'énergie interne du conducteur, et généralement de sa température. \\

L'énergie dissipée sous forme de chaleur entre deux instants $t_1$ et $t_2$ par un dipôle de résistance $R$ lorsque circule un courant d'intensité $i$ s'écrit~:

$$ Q_{joule} = R \int_{t_1}^{t_2} i^2\,dt $$

La puissance moyenne s'écrit alors~:

$$ P = \dfrac{Q_{joule}}{t_2 - t_1} = \dfrac{R}{t_2 - t_1} \int_{t_1}^{t_2}i^2\,dt $$\\

Dans le cas d'un courant \underline{constant} $I$, l'expression devient alors~:

\begin{equation}
	P = R\,I^2
\end{equation}

\subsubsection*{Résistance équivalente}

La \textbf{résistance équivalente} consiste à remplacer dans une partie du circuit un ensemble de résistances par une seule, qui doit être équivalente (dans le sens où le comportement du circuit doit être le même).\\ 

\begin{itemize}
\item \underline{Association en série} : La résistance du dipôle équivalent vaut la somme des résistances de chacun des dipôles. \\

\begin{center}
\begin{minipage}{.2\textwidth}
\begin{center}
\includesvg{part01/chap03/r_serie}
\end{center}
\end{minipage}
\hspace{1cm}
\begin{minipage}{.3\textwidth} 
\begin{equation}
	R_{eq} = R_1 + R_2
\end{equation}
\end{minipage}
\end{center}

\vspace{0.5cm}

\item \underline{Association en parallèle} : l'inverse de la résistance du dipôle équivalent vaut la somme des inverses des résistances de chacun des dipôles. \\

\begin{center}
\begin{minipage}{.2\textwidth}
\begin{center}
	\includesvg{part01/chap03/r_parallele}
\end{center}
\end{minipage}
\hspace{1cm}
\begin{minipage}{.3\textwidth} 
\begin{equation}
	\dfrac{1}{R_{eq}} = \dfrac{1}{R_1} + \dfrac{1}{R_2} 
\end{equation}
\end{minipage}
\end{center}
\end{itemize}

\subsection{ L'inductance }

\subsection{ La capacité }
