\chapter{ Analyse temporelle }

\section{Charge d'un condensateur}

Afin d'étudier la charge d'un condensateur, on utilise le montage RC suivant :

\begin{center}
\includesvg{part01/chap05/charge_c}
\end{center}

À l'instant $t=0$, la tension $U_{in}$ passe de 0V à $E_0$. \\

La loi des mailles donne :

$$U_{in} = U_R + U_C $$

Avec pour la résistance et pour le condensateur :

$$U_R = R\,I$$
$$ I = C \, \dfrac{d\,U_C}{dt} $$

Ce qui permet d'établir l'équation différentielle suivante pour $t>0$~:

$$ RC\,\dfrac{dU_C}{dt} + U_C = E_0 $$ \\

On introduit dans cette expression la constante de temps $\tau = RC$

La solution de l'équation différentielle (en tenant compte des conditions aux limites) est alors~:

$$ U_C(t) = E_0\,(\,1 - e^{-t/\tau}\,) $$

\begin{figure}[!h]
\begin{minipage}{13cm}
\begin{center}
\begin{gnuplot}[terminal=epslatex, terminaloptions=color dashed]
set key at 3,2 horizontal center
set key width 2
set key opaque
set sample 1000
set arrow from graph 0,1 to graph 0,1.05 size screen 0.025,15,60 filled ls -1
set xzeroaxis linetype -1
set xtics axis 
set ytics border nomirror
set border 2
set xr [-1.5:5]
set yr [-1:10]
set xlabel 't (s)'
set ylabel 'Volts'
set tmargin 1
set rmargin 2
R = 2000
C = 400e-6
Tau = R*C
E = 8
step(x) = x>0 ? E : 0
reponse(x) = x<0 ? 0 : E*(1-exp(-x/Tau)) 
plot step(x) w l lc 1 lw 3 t '$U_{in}$',\
reponse(x) w l lc 2 lw 3 t '$U_C$'
\end{gnuplot}
\end{center}
\end{minipage}
\begin{minipage}{3cm}
$E_0 = 8V$ \\
$R = 2\,k\Omega$ \\
$C = 400 \, \mu F$ \\
\bigskip
$ \tau = 0.8\,s $ \\
\end{minipage} \\
	\caption{Charge d'un condensateur}
\end{figure}

\subsection*{Influence de la constante de temps}
\smallskip
\begin{itemize}

\item Au bout d'un temps $3\tau$, la charge du condensateur est à $95\%$ de la valeur finale.

$$ U(3\tau) = E_0\,(\,1-e^{-3}_,) = 0.95 \, E_0$$ \\

\item Au bout d'un temps $5\tau$, la charge du condensateur est à $99.3\%$ de la valeur finale.

$$ U(5\tau) = E_0\,(\,1-e^{-5}_,) = 0.993 \, E_0$$ \\


\item La tangente à l'origine de la courbe coupe la valeur limite ($E_0$) en $t = \tau$

\end{itemize}

\pagebreak

\section{Décharge d'un condensateur}

Afin d'étudier la décharge d'un condensateur, on utilise le montage RC suivant :
\begin{center}
\includesvg{part01/chap05/decharge_c}
\end{center}
À l'instant $t=0$, la tension $U_{in}$ passe de $E_0$ à $0V$. La loi des mailles donne :
\begin{center}
$U_{in} = U_R + U_C$ avec $U_R = R\,I$ et $ I = C \, \dfrac{d\,U_C}{dt} $
\end{center}
Ce qui permet d'établir l'équation différentielle suivante pour $t>0$~:
$$ RC\,\dfrac{dU_C}{dt} + U_C = 0 $$ 
En posant $\tau = RC$, la solution de l'équation différentielle est alors~: $ U_C(t) = E_0\,e^{-t/\tau} $

\begin{figure}[!h]
\begin{minipage}{13cm}
\begin{center}
\begin{gnuplot}[terminal=epslatex, terminaloptions=color dashed]
set key at 3,2 horizontal center
set key width 2
set key opaque
set sample 1000
set arrow from graph 0,1 to graph 0,1.05 size screen 0.025,15,60 filled ls -1
set xzeroaxis linetype -1
set xtics axis 
set ytics border nomirror
set border 2
set xr [-1.5:5]
set yr [-1:10]
set xlabel 't (s)'
set ylabel 'Volts'
set tmargin 1
set rmargin 2
R = 2000
C = 400e-6
Tau = R*C
E = 8
step(x) = x>0 ? 0 : E
reponse(x) = x<0 ? E : E*exp(-x/Tau)
plot step(x) w l lc 1 lw 3 t '$U_{in}$',\
reponse(x) w l lc 2 lw 3 t '$U_C$'
\end{gnuplot}
\end{center}
\end{minipage}
\begin{minipage}{3cm}
$E_0 = 8V$ \\
$R = 2\,k\Omega$ \\
$C = 400 \, \mu F$ \\
\bigskip
$ \tau = 0.8\,s $ \\
\end{minipage} \\
	\caption{décharge d'un condensateur}
\end{figure}

\subsection*{Influence de la constante de temps}
\smallskip
\begin{itemize}

\item Au bout d'un temps $3\tau$, la charge du condensateur est à $5\%$ de la valeur initiale.

$$ U(3\tau) = E_0\,e^{-3} = 0.05 \, E_0$$ \\

\item Au bout d'un temps $5\tau$, la charge du condensateur est à $0.6\%$ de la valeur initiale.

$$ U(5\tau) = E_0\,e^{-5} = 0.006 \, E_0$$ \\

\item La tangente à l'origine de la courbe coupe la valeur 0 en $t = \tau$

\end{itemize}

\section{Établissement du courant dans une inductance}

On considère le schéma du circuit RL suivant :

\begin{center}
\includesvg{part01/chap05/echelon_l}
\end{center}

À l'instant $t=0$, la tension $U_{in}$ passe de 0V à $E_0$. \\

La loi des mailles donne :

$$U_{in} = U_R + U_L $$

Avec pour la résistance et pour le condensateur :

$$U_R = R\,I$$
$$ U_L = L \, \dfrac{d\,I}{dt} $$

Ce qui permet d'établir l'équation différentielle suivante pour $t>0$~:

$$ R\,I + L\,\dfrac{d\,I}{dt}   = E_0 $$ \\

On introduit dans cette expression la constante de temps $\tau = \dfrac{L}{R}$~:

$$ I + \dfrac{L}{R}\dfrac{d\,I}{dt}   = \dfrac{E_0}{R} $$ \\

Pour $t>0$, la solution de cette équation différentielle est~:

$$ I(t) = \dfrac{E_0}{R}\,(\,1 - e^{-t/\tau}\,) $$

Ce qui correspond à la tension $U_L$ suivante :

$$ U_L(t) = L \dfrac{d\,I}{d\,t} = E_0 \, e^{-t/\tau}\ $$


\begin{figure}[!h]
	\begin{center}
	\begin{gnuplot}[terminal=epslatex, terminaloptions={color dashed}]
set key horizontal outside bottom
set key width 2
set key opaque
set sample 1000
set arrow from graph 0,1 to graph 0,1.05 size screen 0.025,15,60 filled ls -1
set arrow from graph 1,1 to graph 1,1.05 size screen 0.025,15,60 filled ls -1
set xzeroaxis linetype -1
set xtics axis 
set ytics  border nomirror
set y2tics border nomirror
set border 10
set xr [-0.5:5.9]
set link y2 via y/10 inverse y*10
set yr [-1:10]
set xlabel 't (s)'
set ylabel 'Volts'
set y2label 'Ampères'
set tmargin 1
set rmargin 2
R = 8
L = 10
Tau = L/R
E = 8
step(x) = x>0 ? E : 0
tension(x) = x<0 ? 0 : E*exp(-x/Tau)
courant(x) = x<0 ? 0 : E/R*(1-exp(-x/Tau)) 
plot step(x) w l lc 1 lw 3 t '$U_{in}$',\
tension(x) w l lc 2 lw 3 t '$U_L$',\
courant(x) w l lc 3 lw 3 t '$I_L$' axis x1y2
\end{gnuplot} 
\begin{minipage}{\textwidth}
\bigskip
\begin{center}
$E_0 = 8V \quad R = 8\,\Omega \quad L = 10\,H \quad \tau = 1.25\,s $ 
\end{center}
\end{minipage}
\end{center}
\caption{Etablissement du courant dans un circuit RL}
\end{figure}

Les équations différentielles étant les mêmes, les remarques concernant la constante de temps $\tau$ effectuées dans les chapitres précédants restent valides. \\


\section{Rupture du courant dans une inductance}

On considère le circuit RL suivant :

\begin{center}
\includesvg{part01/chap05/echelon2_l}
\end{center}

À l'instant $t=0$, la tension $U_{in}$ passe de $E_0$ à $0V$. \\

La loi des mailles donne :

$$U_{in} = U_R + U_L $$

Avec pour la résistance et pour le condensateur :

$$U_R = R\,I$$
$$ U_L = L \, \dfrac{d\,I}{dt} $$ \\

Ce qui permet d'établir l'équation différentielle suivante pour $t>0$~:

$$ R\,I + L\,\dfrac{d\,I}{dt}   = 0 $$ 

On introduit dans cette expression la constante de temps $\tau = \dfrac{L}{R}$~:

$$ I + \dfrac{L}{R}\dfrac{d\,I}{dt}   = 0 $$ \\

Pour $t>0$, la solution de cette équation différentielle est~:

$$ I(t) = \dfrac{E_0}{R}\,e^{-t/\tau} $$  

\pagebreak

Ce qui correspond à la tension $U_L$ suivante :

$$ U_L(t) = L \dfrac{d\,I}{d\,t} = - E_0 \, e^{-t/\tau}\ $$

\begin{figure}[!h]
	\begin{center}
	\begin{gnuplot}[terminal=epslatex, terminaloptions={color dashed}]
set key horizontal outside bottom
set key width 2
set key opaque
set sample 1000
set arrow from graph 0,1 to graph 0,1.05 size screen 0.025,15,60 filled ls -1
set arrow from graph 1,1 to graph 1,1.05 size screen 0.025,15,60 filled ls -1
set xzeroaxis linetype -1
set xtics axis 
set ytics  border nomirror
set y2tics border nomirror
set border 10
set xr [-0.5:5.9]
set link y2 via y/10 inverse y*10
set yr [-10:10]
set xlabel 't (s)'
set ylabel 'Volts'
set y2label 'Ampères'
set tmargin 1
set rmargin 2
R = 8
L = 10
Tau = L/R
E = 8
step(x) = x>0 ? 0 : E
tension(x) = x<0 ? 0 : -E*exp(-x/Tau)
courant(x) = x<0 ? E/R : E/R*exp(-x/Tau)
plot step(x) w l lc 1 lw 3 t '$U_{in}$',\
tension(x) w l lc 2 lw 3 t '$U_L$',\
courant(x) w l lc 3 lw 3 t '$I_L$' axis x1y2
\end{gnuplot} 
\begin{minipage}{\textwidth}
\bigskip
\begin{center}
$E_0 = 8V \quad R = 8\,\Omega \quad L = 10\,H \quad \tau = 1.25\,s $ 
\end{center}
\end{minipage}
\end{center}
\caption{Rupture du courant dans un circuit RL}
\end{figure}

Les équations différentielles étant les mêmes, les remarques concernant la constante de temps $\tau$ effectuées dans les chapitres précédants restent valides. \\

\subsection*{Note:}

Dans le cas présenté ci-dessus, le courant n'est pas brutalement interrompu. Considérer $U_{in} = 0V$ signifie que l'on court-circuite l'alimentation, et donc qu'un chemin est disponible pour qu'un courant puisse s'établir. \\

Lorsque ce n'est pas le cas (ouverture du circuit), le courant passe brutalement de $E_0 / R$ à $0$. Sa dérivée en $t=0$ est donc très grande (théoriquement infinie). La tension aux bornes de l'inductance peut alors s'avérer très (voir trop) importante.  Nous verrons dans les chapitres suivants que cela justifie l'usage d'une diode de roue libre.\\

\section{Réponse à un échelon du circuit RLC série}

On considère le circuit RLC suivant~:

\begin{center}
\includesvg{part01/chap05/echelon_RLC_Serie}
\end{center}

La loi des mailles donne la relation suivante :

$$ U_{in} = U_R + U_L + U_C $$

Avec pour la résistance, l'inductance et pour la capacité :

$$ U_R = R\,I \quad U_L = L\,\dfrac{d\,I}{d\,t} \quad I = C \, \dfrac{d\,U_C}{d\,t}$$ 

Pour $t>0$, on obtient l'équation différentielle suivante~:

$$ LC\,\dfrac{d^{2}\,U_C}{d\,t^2} + RC\,\dfrac{d\,U_C}{d\,t} + U_C = E_0 $$

On définit alors les constantes suivantes~:\\

\begin{itemize}
\item \textbf{La pulsation propre} ($\omega_0$)

	$$ \omega_0 = \dfrac{1}{\sqrt{LC}} $$

\item \textbf{Le coefficient d'amortissement} ($\lambda$)

	$$ \lambda = \dfrac{R}{2\,L\,\omega_0} $$\\

\end{itemize}

L'équation du circuit devient alors~:

$$ \dfrac{d^{2}\,U_C}{d\,t^2} + 2\,\lambda\,\omega_0\,\dfrac{d\,U_C}{d\,t} + \omega_0^2\,U_C = \omega_0^2\,E_0 $$\\

On utilise la méthode de résolution d'une EDO d'ordre 2 présentée en annexe XX.

\begin{enumerate}

\item \textbf{Solution particulière~:} 
	$$U_C(t) = constante = E_0$$ 

\item \textbf{Solution de l'équation sans second membre~:}

$$ \dfrac{d^{2}\,U_C}{d\,t^2} + 2\,\lambda\,\omega_0\,\dfrac{d\,U_C}{d\,t} + w_0^2\,U_C = 0 $$

		Calcul du déterminant~: $\Delta = (2\lambda\omega_0)^2 - 4 * w_0^2 = 4\, \omega_0^2\,(\lambda^2 - 1 )$ 

	Le signe de $\Delta $ dépend de $\lambda $. Il y a alors 3 cas possibles~:\\

	\begin{itemize}
		\item \textbf{Le régime apériodique~:} ($\Delta > 0 $) \\

			Le polynôme caractéristique admet deux solutions~:
			$$ r_1 = \dfrac{ -2\,\lambda\,\omega_0 - \sqrt{\Delta}}{ 2 } \quad et \quad	r_2 = \dfrac{ -2\,\lambda\,\omega_0 + \sqrt{\Delta}}{ 2 }  $$

			et la solution de l'équation sans second membre est de la forme~:\\
			\begin{center}
				$U_C(t) = A\,e^{r_1\,t} + B\,e^{r_2\,t}$ (avec A et B deux constantes réelles)
			\end{center}
		\bigskip
		\item \textbf{Le régime critique~:} ($\Delta = 0$)\\	

			Le polynôme caractéristique admet une racine double et la solution de l'équation sans second membre est de la forme~:\\

			\begin{center}
			$U_C(t) = (\,A + B\,t\,)\,e^{ -w_0\,t} $ avec A et B deux constantes réelles. \\
			\end{center}
					
		\bigskip
		\item \textbf{Le régime pseudo-périodique~:} ($\Delta < 0 $)\\	

			Le polynôme caractéristique admet deux solutions complexes conjuguées~:
			$$ r_1 =  -\,\lambda\,\omega_0 - j\,\dfrac{\sqrt{|\Delta|}}{ 2 } \quad et \quad	r_2 = -\,\lambda\,\omega_0 + j\,\dfrac{\sqrt{|\Delta|}}{ 2 }  $$

			et la solution de l'équation sans second membre est de la forme~:\\
				$$U_C(t) = e^{-\dfrac{R}{2L}}\,(A\,cos(\omega_0\,t)\,+\,B\,sin(\omega_0\,t))$$

				avec A et B deux constantes réelles.

		\end{itemize}
	\bigskip
	\item \textbf{Ecriture de la solution générale~:}\\

Pour obtenir la solution générale à l'équation du circuit, on ajoute la solution particulière et la solution de l'équation sans second membre. Les valeurs des constantes A et B sont obtenues par l'étude de la tension $U_C$ et du courant aux instants $t=0$ et $t=\infty$.

\end{enumerate}

