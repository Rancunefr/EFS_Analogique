\chapter{Charge d'un condensateur}

Afin d'étudier la charge d'un condensateur, on utilise le montage suivant :

\begin{center}
\includesvg{part02/chap01/charge_c}
\end{center}

À l'instant $t=0$, la tension $U_{in}$ passe de 0V à $E_0$. \\

La loi des mailles donne :

$$U_{in} = U_R + U_C $$

Avec pour la résistance et pour le condensateur :

$$U_R = R\,I$$
$$ I = C \, \dfrac{d\,U_C}{dt} $$

Ce qui permet d'établir l'équation différentielle suivante pour $t>0$~:

$$ RC\,\dfrac{dU_C}{dt} + U_C = E_0 $$ \\

On introduit dans cette expression la constante de temps $\tau = RC$

La solution de l'équation différentielle (en tenant compte des conditions aux limites) est alors~:

$$ U_C(t) = E_0\,(\,1 - e^{-t/\tau}\,) $$


\begin{minipage}{13cm}
\begin{center}
\begin{gnuplot}[terminal=epslatex, terminaloptions=color dashed]
set key at 3,2 horizontal center
set key width 2
set key opaque
set sample 1000
set xzeroaxis
set xtics axis 
set ytics border nomirror
set border 2
set xr [-1:5]
set yr [-1:10]
set xlabel 't (s)'
set ylabel 'Volts'
set tmargin 1
set rmargin 2
R = 2000
C = 400e-6
Tau = R*C
E = 8
step(x) = x>0 ? E : 0
reponse(x) = x<0 ? 0 : E*(1-exp(-x/Tau)) 
plot step(x) w l lc 1 lw 3 t '$U_{in}$',\
reponse(x) w l lc 2 lw 3 t '$U_C$'
\end{gnuplot}
\end{center}
\end{minipage}
\begin{minipage}{3cm}
$E_0 = 8V$ \\
$R = 2\,k\Omega$ \\
$C = 400 \, \mu F$ \\
\bigskip
$ \tau = 0.8\,s $ \\
\end{minipage}
