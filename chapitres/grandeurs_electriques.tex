\chapter{Les grandeurs électriques}

\section{Le Potentiel}

En tout point de l'espace est défini un \textbf{potentiel}. Cette valeur scalaire correspond à l'énergie potentielle électrostatique que posséderait une charge électrique unitaire située en ce point. \\

\begin{tabular}{ll}
\textbf{Notation usuelle~:} & V \\
\textbf{Unité~:} & Volt \\
\textbf{Unité SI~:} & ${kg} \cdot m^2 \cdot {s}^{-3} \cdot A^{-1}$ \\
\textbf{Nature~:} & Grandeur scalaire \\
\end{tabular}

\section{La tension}

Une \textbf{tension} est la circulation du champs électrique le long d'un circuit. Dans le cas du régime stationnaire, cette notion se confond avec celle de \textbf{"différence de potentiel"} entre les deux extrémités du circuit. \\

\begin{tabular}{ll}
\textbf{Notation usuelle~:} & $U$, $U_{AB}$ \\
\textbf{Unité~:} & Volt \\
\textbf{Unité SI~:} & ${kg} \cdot m^2 \cdot {s}^{-3} \cdot A^{-1}$ \\
\textbf{Nature~:} & Grandeur scalaire \\
\end{tabular} \\

En électronique, une tension se mesure toujours \emph{entre deux points}, généralement à l'aide d'un multimètre ou d'un oscilloscope. \\

\textbf{ Représentation graphique: } \\

\includesvg{figures/tension}

test

\section{Le courant}

\section{La puissance}

\section{L'Energie electrique}
